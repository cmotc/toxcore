Every function that can fail takes a function-\/specific error code pointer that can be used to diagnose problems with the Tox state or the function arguments. The error code pointer can be N\+U\+L\+L, which does not influence the function's behaviour, but can be done if the reason for failure is irrelevant to the client.

The exception to this rule are simple allocation functions whose only failure mode is allocation failure. They return N\+U\+L\+L in that case, and do not set an error code.

Every error code type has an O\+K value to which functions will set their error code value on success. Clients can keep their error code uninitialised before passing it to a function. The library guarantees that after returning, the value pointed to by the error code pointer has been initialised.

Functions with pointer parameters often have a N\+U\+L\+L error code, meaning they could not perform any operation, because one of the required parameters was N\+U\+L\+L. Some functions operate correctly or are defined as effectless on N\+U\+L\+L.

Some functions additionally return a value outside their return type domain, or a bool containing true on success and false on failure.

All functions that take a Tox instance pointer will cause undefined behaviour when passed a N\+U\+L\+L Tox pointer.

All integer values are expected in host byte order.

Functions with parameters with enum types cause unspecified behaviour if the enumeration value is outside the valid range of the type. If possible, the function will try to use a sane default, but there will be no error code, and one possible action for the function to take is to have no effect. 