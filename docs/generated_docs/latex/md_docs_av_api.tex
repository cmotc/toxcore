\#\+A/\+V A\+P\+I reference

\subsection*{Take toxmsi/phone.\+c as a reference}

\subsubsection*{Initialization\+:}

``` phone\+\_\+t$\ast$ init\+Phone(uint16\+\_\+t \+\_\+listen\+\_\+port, uint16\+\_\+t \+\_\+send\+\_\+port); ```

function initializes sample phone. \+\_\+listen\+\_\+port and \+\_\+send\+\_\+port are variables only meant for local testing. You will not have to do anything regarding to that since everything will be started within a mesenger.

Phone requires one msi session and two rtp sessions ( one for audio and one for video ).

``` msi\+\_\+session\+\_\+t$\ast$ msi\+\_\+init\+\_\+session( void$\ast$ \+\_\+core\+\_\+handler, const uint8\+\_\+t$\ast$ \+\_\+user\+\_\+agent ); ```

initializes msi session. Params\+:

``` void$\ast$ \+\_\+core\+\_\+handler -\/ pointer to an object handling networking, const uint8\+\_\+t$\ast$ \+\_\+user\+\_\+agent -\/ string describing phone client version. ```

Return value\+: msi\+\_\+session\+\_\+t$\ast$ -\/ pointer to a newly created msi session handler.

\subsubsection*{msi\+\_\+session\+\_\+t reference\+:}

How to handle msi session\+: Controlling is done via callbacks and action handlers. First register callbacks for every state/action received and make sure N\+O\+T T\+O P\+L\+A\+C\+E S\+O\+M\+E\+T\+H\+I\+N\+G L\+I\+K\+E L\+O\+O\+P\+S T\+H\+A\+T T\+A\+K\+E\+S A L\+O\+T O\+F T\+I\+M\+E T\+O E\+X\+E\+C\+U\+T\+E; every callback is being called directly from event loop. You can find examples in phone.\+c.

Register callbacks\+: ``` void msi\+\_\+register\+\_\+callback\+\_\+call\+\_\+started ( M\+C\+A\+L\+L\+B\+A\+C\+K ); void msi\+\_\+register\+\_\+callback\+\_\+call\+\_\+canceled ( M\+C\+A\+L\+L\+B\+A\+C\+K ); void msi\+\_\+register\+\_\+callback\+\_\+call\+\_\+rejected ( M\+C\+A\+L\+L\+B\+A\+C\+K ); void msi\+\_\+register\+\_\+callback\+\_\+call\+\_\+ended ( M\+C\+A\+L\+L\+B\+A\+C\+K );

void msi\+\_\+register\+\_\+callback\+\_\+recv\+\_\+invite ( M\+C\+A\+L\+L\+B\+A\+C\+K ); void msi\+\_\+register\+\_\+callback\+\_\+recv\+\_\+ringing ( M\+C\+A\+L\+L\+B\+A\+C\+K ); void msi\+\_\+register\+\_\+callback\+\_\+recv\+\_\+starting ( M\+C\+A\+L\+L\+B\+A\+C\+K ); void msi\+\_\+register\+\_\+callback\+\_\+recv\+\_\+ending ( M\+C\+A\+L\+L\+B\+A\+C\+K ); void msi\+\_\+register\+\_\+callback\+\_\+recv\+\_\+error ( M\+C\+A\+L\+L\+B\+A\+C\+K );

void msi\+\_\+register\+\_\+callback\+\_\+requ\+\_\+timeout ( M\+C\+A\+L\+L\+B\+A\+C\+K ); ```

M\+C\+A\+L\+L\+B\+A\+C\+K is defined as\+: void ({\itshape callback) (void} \+\_\+arg) msi\+\_\+session\+\_\+t$\ast$ handler is being thrown as \+\_\+arg so you can use that and \+\_\+agent\+\_\+handler to get to your own phone handler directly from callback.

Actions\+:

``` int msi\+\_\+invite ( msi\+\_\+session\+\_\+t$\ast$ \+\_\+session, call\+\_\+type \+\_\+call\+\_\+type, uint32\+\_\+t \+\_\+timeoutms ); ```

Sends call invite. Before calling/sending invite msi\+\_\+session\+\_\+t\+::\+\_\+friend\+\_\+id is needed to be set or else it will not work. \+\_\+call\+\_\+type is type of the call ( Audio/\+Video ) and \+\_\+timeoutms is how long will poll wait until request is terminated.

``` int msi\+\_\+hangup ( msi\+\_\+session\+\_\+t$\ast$ \+\_\+session ); ``` Hangs up active call

``` int msi\+\_\+answer ( msi\+\_\+session\+\_\+t$\ast$ \+\_\+session, call\+\_\+type \+\_\+call\+\_\+type ); ``` Answer incomming call. \+\_\+call\+\_\+type set's callee call type.

``` int msi\+\_\+cancel ( msi\+\_\+session\+\_\+t$\ast$ \+\_\+session ); ``` Cancel current request.

``` int msi\+\_\+reject ( msi\+\_\+session\+\_\+t$\ast$ \+\_\+session ); ``` Reject incomming call.

\subsubsection*{Now for rtp\+:}

You will need 2 sessions; one for audio one for video. You start them with\+: ``` rtp\+\_\+session\+\_\+t$\ast$ rtp\+\_\+init\+\_\+session ( int \+\_\+max\+\_\+users, int \+\_\+multi\+\_\+session ); ```

Params\+: ``` int \+\_\+max\+\_\+users -\/ max users. -\/1 if undefined int \+\_\+multi\+\_\+session -\/ any positive number means uses multi session; -\/1 if not. ```

Return value\+: ``` rtp\+\_\+session\+\_\+t$\ast$ -\/ pointer to a newly created rtp session handler. ```

\subsubsection*{How to handle rtp session\+:}

Take a look at ``` void$\ast$ phone\+\_\+handle\+\_\+media\+\_\+transport\+\_\+poll ( void$\ast$ \+\_\+hmtc\+\_\+args\+\_\+p ) in phone.\+c ``` on example. Basically what you do is just receive a message via\+: ``` struct rtp\+\_\+msg\+\_\+s$\ast$ rtp\+\_\+recv\+\_\+msg ( rtp\+\_\+session\+\_\+t$\ast$ \+\_\+session ); ```

and then you use payload within the rtp\+\_\+msg\+\_\+s struct. Don't forget to deallocate it with\+: void rtp\+\_\+free\+\_\+msg ( rtp\+\_\+session\+\_\+t$\ast$ \+\_\+session, struct rtp\+\_\+msg\+\_\+s$\ast$ \+\_\+msg ); Receiving should be thread safe so don't worry about that.

When you capture and encode a payload you want to send it ( obviously ).

first create a new message with\+: ``` struct rtp\+\_\+msg\+\_\+s$\ast$ rtp\+\_\+msg\+\_\+new ( rtp\+\_\+session\+\_\+t$\ast$ \+\_\+session, const uint8\+\_\+t$\ast$ \+\_\+data, uint32\+\_\+t \+\_\+length ); ```

and then send it with\+: ``` int rtp\+\_\+send\+\_\+msg ( rtp\+\_\+session\+\_\+t$\ast$ \+\_\+session, struct rtp\+\_\+msg\+\_\+s$\ast$ \+\_\+msg, void$\ast$ \+\_\+core\+\_\+handler ); ```

\+\_\+core\+\_\+handler is the same network handler as in msi\+\_\+session\+\_\+s struct.

\subsection*{A/\+V initialization\+:}

``` int init\+\_\+receive\+\_\+audio(codec\+\_\+state $\ast$cs); int init\+\_\+receive\+\_\+video(codec\+\_\+state $\ast$cs); Initialises the A/\+V decoders. On failure it will print the reason and return 0. On success it will return 1.

int init\+\_\+send\+\_\+audio(codec\+\_\+state $\ast$cs); int init\+\_\+send\+\_\+video(codec\+\_\+state $\ast$cs); Initialises the A/\+V encoders. On failure it will print the reason and return 0. On success it will return 1. init\+\_\+send\+\_\+audio will also let the user select an input device. init\+\_\+send\+\_\+video will determine the webcam's output codec and initialise the appropriate decoder.

int video\+\_\+encoder\+\_\+refresh(codec\+\_\+state $\ast$cs, int bps); Reinitialises the video encoder with a new bitrate. ffmpeg does not expose the needed V\+P8 feature to change the bitrate on the fly, so this serves as a workaround. In the future, V\+P8 should be used directly and ffmpeg should be dropped from the dependencies. The variable bps is the required bitrate in bits per second. ```

\subsubsection*{A/\+V encoding/decoding\+:}

``` void $\ast$encode\+\_\+video\+\_\+thread(void $\ast$arg); ``` Spawns the video encoding thread. The argument should hold a pointer to a codec\+\_\+state. This function should only be called if video encoding is supported (when init\+\_\+send\+\_\+video returns 1). Each video frame gets encoded into a packet, which is sent via R\+T\+P. Every 60 frames a new bidirectional interframe is encoded. ``` void $\ast$encode\+\_\+audio\+\_\+thread(void $\ast$arg); ``` Spawns the audio encoding thread. The argument should hold a pointer to a codec\+\_\+state. This function should only be called if audio encoding is supported (when init\+\_\+send\+\_\+audio returns 1). Audio frames are read from the selected audio capture device during intitialisation. This audio capturing can be rerouted to a different device on the fly. Each audio frame is encoded into a packet, and sent via R\+T\+P. All audio frames have the same amount of samples, which is defined in A\+V\+\_\+codec.\+h. ``` int video\+\_\+decoder\+\_\+refresh(codec\+\_\+state $\ast$cs, int width, int height); ``` Sets the S\+D\+L window dimensions and creates a pixel buffer with the requested size. It also creates a scaling context, which will be used to convert the input image format to Y\+U\+V420\+P.

``` void $\ast$decode\+\_\+video\+\_\+thread(void $\ast$arg); ``` Spawns a video decoding thread. The argument should hold a pointer to a codec\+\_\+state. The codec\+\_\+state is assumed to contain a successfully initialised video decoder. This function reads video packets and feeds them to the video decoder. If the video frame's resolution has changed, video\+\_\+decoder\+\_\+refresh() is called. Afterwards, the frame is displayed on the S\+D\+L window. ``` void $\ast$decode\+\_\+audio\+\_\+thread(void $\ast$arg); ``` Spawns an audio decoding thread. The argument should hold a pointer to a codec\+\_\+state. The codec\+\_\+state is assumed to contain a successfully initialised audio decoder. All received audio packets are pushed into a jitter buffer and are reordered. If there is a missing packet, or a packet has arrived too late, it is treated as a lost packet and the audio decoder is informed of the packet loss. The audio decoder will then try to reconstruct the lost packet, based on information from previous packets. Audio is played on the default Open\+A\+L output device.

If you have any more qustions/bug reports/feature request contact the following users on the irc channel \#tox-\/dev on irc.\+freenode.\+net\+: For R\+T\+P and M\+S\+I\+: mannol For audio and video\+: Martijnvdc 